\documentclass[9pt]{book}
%\usepackage[spanish]{babel}
\usepackage{fancyhdr}
\usepackage{color}
\usepackage[latin1]{inputenc}
\usepackage{indentfirst}
\usepackage{graphicx}
%\usepackage{rotating}
\usepackage{amssymb}
\usepackage{amsmath}
%\usepackage{shapepar}
%\usepackage[small]{caption}
%\usepackage{lscape}
%\usepackage{natbib}
\pagestyle{fancy}
%\usepackage{mathrsfs}
%\usepackage{multirow}
\newcommand{\ihat}{\mathbf {\hat \imath}}
\newcommand{\jhat}{\mathbf {\hat \jmath}}

%\renewcommand{\chaptermark}[1]{\markboth{#1}{}}
%\renewcommand{\sectionmark}[1]{\markright{\thesection\ #1}}
%\lhead[\fancyplain{}{\bfseries\thepage}] {\fancyplain{}{\bfseries\rightmark}}
%\rhead[\fancyplain{}{\bfseries\leftmark}] {\fancyplain{}{\bfseries\thepage}}
%\cfoot{}
%\newcommand{\ud}{\;\mathrm{d}}
%\renewcommand{\sin}{\mbox{\,sen\,}}
%\renewcommand{\tablename}{Tabla}
%\setlength{\captionmargin}{19pt}
%\renewcommand{\captionlabelfont}{\sffamily}
%\renewcommand{\labelenumi}{(\arabic{enumi})}
% Para que no ponga encabezados en hojas vacías
%\newcommand{\clearemptydoublepage}{\newpage{\pagestyle{empty}
%            \cleardoublepage}}
% Para que incluya en el índice resumen, introducción, etc.
%\newcommand{\capitulo}[1]{\chapter*{#1}
%              \addcontentsline{toc}{chapter}{#1}
%               \markboth{\MakeUppercase { #1 } }{\MakeUppercase { #1 } }}



\begin{document}

\numberwithin{equation}{chapter}
  \renewcommand{\thepage}{\arabic{page}}
  \setcounter{page}{1}
  \renewcommand{\tablename}{Tabla}

 \chapter{Introducci\'on}
Introducci\'on
\section{Estructuras de micro-escala en la cromosfera solar}
observaciones de la red cromosfericas y del experimento VAULT.
\section{El c\'odigo \emph{PakalMPI}}
Descripci\'on del c\'odigo, puntualmente hay qu\'e explicar primero para qu\'e sirve actualmente y a partir de ello, plantear las modificaciones.


\chapter{Emisi\'on submilim\'etrica solar}
Mecanismos de la emisi\'on milim\'etrica del plasma del sol quieto, antecedente hist\'orico y enfoque del problema a resolver (qu\'e se va a hacer, por qu\'e se va a hacer y c\'omo se va a a hacer)
  
\chapter{Modelo de Micro estrucras magneticas solares}
\section{Caracteristicas Fisicas}
Aqui van las propiedades fisicas de los modelos que vamos a implementar. Por ejemplo, definir las escalas de altura, las intensidades de campo, temperatura, densidades, etc.
Para llevar a cabo la simulaci\'on del comportamiento de los campos magn\'eticos se utilizaron dos modelos distintos; uno en forma de arcos magn\'eticos \cite{loops} y otro en forma de un flujo emergente \cite{flujoemergente}

\section{Implementaci\'on del modulo para Magnetohidrost\'atica}
Aqui va la info de como implementaste los modelos.


\chapter{Resultados}
\section{Micro arco magnetico}
resultados de las simulaciones para el modelo de arcos magneticos
\section{Flujo emergente}
resultados de las simulaciones para flujo emergente


\chapter{Discusi\'on y conclusiones}
Comparaci\'on entre ambos modelos, discusion y resultados
hola Springer Science \& Business Media, Aug 26, 2006

\begin{thebibliography}{9}

\bibitem{loops} 
Markus Aschwanden. 
\textit{Physics of the Solar Corona} 
Springer Science \& Business Media, 2006.

\bibitem{flujoemergente} 
Corrigeme. 
\textit{Corrigeme} 
Text here


\end{thebibliography}


\end{document}
