\documentclass[9pt]{book}
%\usepackage[spanish]{babel}
\usepackage{fancyhdr}
\usepackage{color}
\usepackage[latin1]{inputenc}
\usepackage{indentfirst}
\usepackage{graphicx}
%\usepackage{rotating}
\usepackage{amssymb}
\usepackage{amsmath}
%\usepackage{shapepar}
%\usepackage[small]{caption}
%\usepackage{lscape}
%\usepackage{natbib}
\pagestyle{fancy}
%\usepackage{mathrsfs}
%\usepackage{multirow}
\newcommand{\ihat}{\mathbf {\hat \imath}}
\newcommand{\jhat}{\mathbf {\hat \jmath}}

%\renewcommand{\chaptermark}[1]{\markboth{#1}{}}
%\renewcommand{\sectionmark}[1]{\markright{\thesection\ #1}}
%\lhead[\fancyplain{}{\bfseries\thepage}] {\fancyplain{}{\bfseries\rightmark}}
%\rhead[\fancyplain{}{\bfseries\leftmark}] {\fancyplain{}{\bfseries\thepage}}
%\cfoot{}
%\newcommand{\ud}{\;\mathrm{d}}
%\renewcommand{\sin}{\mbox{\,sen\,}}
%\renewcommand{\tablename}{Tabla}
%\setlength{\captionmargin}{19pt}
%\renewcommand{\captionlabelfont}{\sffamily}
%\renewcommand{\labelenumi}{(\arabic{enumi})}
% Para que no ponga encabezados en hojas vacías
%\newcommand{\clearemptydoublepage}{\newpage{\pagestyle{empty}
%            \cleardoublepage}}
% Para que incluya en el índice resumen, introducción, etc.
%\newcommand{\capitulo}[1]{\chapter*{#1}
%              \addcontentsline{toc}{chapter}{#1}
%               \markboth{\MakeUppercase { #1 } }{\MakeUppercase { #1 } }}



\begin{document}

\numberwithin{equation}{chapter}
  \renewcommand{\thepage}{\arabic{page}}
  \setcounter{page}{1}
  \renewcommand{\tablename}{Tabla}

 \chapter{Introducci\'on}
Introducci\'on
\section{Estructuras de micro-escala en la cromosfera solar}
observaciones de la red cromosfericas y del experimento VAULT.
\section{El c\'odigo \emph{PakalMPI}}
Descripci\'on del c\'odigo, puntualmente hay qu\'e explicar primero para qu\'e sirve actualmente y a partir de ello, plantear las modificaciones.


\chapter{Emisi\'on submilim\'etrica solar}
Mecanismos de la emisi\'on milim\'etrica del plasma del sol quieto, antecedente hist\'orico y enfoque del problema a resolver (qu\'e se va a hacer, por qu\'e se va a hacer y c\'omo se va a a hacer)
  
\chapter{Modelo de Micro estrucras magneticas solares}
\section{Caracteristicas Fisicas}
Aqui van las propiedades fisicas de los modelos que vamos a implementar. Por ejemplo, definir las escalas de altura, las intensidades de campo, temperatura, densidades, etc.

Para llevar a cabo la simulaci\'on del comportamiento de los campos magn\'eticos se utilizaron dos modelos distintos; uno en forma de arcos magn\'eticos \cite{loops} y otro en forma de un flujo emergente \cite{flujoemergente}.

La primer aproximaci\'on para el modelo de campo magn\'etico fue el de un loop con forma semi-circular, el cu\'al puede ser modelado como un campo magn\'etico dipolo generado por un cable dipolo enterrado debajo de la fo\'sfera.
En este modelo es posible calcular las tres componentes del campo magn\'etico por medio de las siguientes ecuaciones \citation{Jackson}:

\begin{equation}
    B_r(r,\theta)=\frac{2mcos\theta}{r^3}
\end{equation}
  
\begin{equation}
    B_\theta(r,\theta)=\frac{2msin\theta}{r^3}
\end{equation}

\begin{equation}
    B_\phi(r,\theta)=0
\end{equation}

Donde m representa el momento magn\'etico inducido por el anillo de corriente I con un conductor de radio a, y los par\'ametros theta y r son los par\'ametros libres que movemos para ir generando los valores del campo.

Para la segunda aproximaci\'on del modelo se consider\'o un flujo emergente, ya que una gran cantidad del flujo magn\'etico en la fot\'osfera solar consiste en fuertes campos en elementos no resueltos. Este modelo consiste en un elemento magn\'etico con una simetr\'ia de un tubo de flujo cil\'indrico incrustado en la fot\'osfera.

Los valores de sus componentes son calculados como

\begin{equation}
B_z(r,z)=B_z(0,z)D(\alpha)
\end{equation}

\begin{equation}
B_r(r,z)=-\frac{1}{2}r\frac{dB_z(0,z)}{dz}D(\alpha)
\end{equation}

Este campo satisface gradiente $ \nabla \cdot B = 0 $. La funci\'on radial $D(\alpha)$ es tomada como
\begin{equation}
 D(\alpha) = 
    \begin{cases}
        (1-alpha^2)^2 & \quad \alpha <= 1 \\
        0   & \quad \alpha > 1
    \end{cases}
\end{equation}

\begin{equation} \label{r_flujo_emergente}
r_0^-2(z) = \frac{2\pi B_z(0,z)}{\phi} \int_{0}^{1} \alpha D(\alpha) d\alpha = \frac{\pi B_z(0,z)}{3 \phi}
\end{equation}

donde $\phi$ es el flujo magn\'etico dentro del tubo.
El campo axial es considerado

\begin{equation}
B_z(0,z)=B_0e^{\frac{-z}{h}}
\end{equation}

Este campo se considera $B_0$ al origen $z = 0$ que fue considerado al nivel de la fot\'osfera (CORRIGEME). Este modelo permite al campo cambiar de acuerdo a la escala de altura h.
Conforme $h\rightarrow \infty , B_z(0,z) \rightarrow B_0$, $a$ constante, y por la ecuaci\'on \ref{r_flujo_emergente} $r_0(z) \rightarrow r_0 = (3\phi / \pi B_0)^\frac{1}{2}$, que es tambi\'en una constante.



\section{Implementaci\'on del modulo para Magnetohidrost\'atica}
Aqui va la info de como implementaste los modelos.


\chapter{Resultados}
\section{Micro arco magnetico}
resultados de las simulaciones para el modelo de arcos magneticos
\section{Flujo emergente}
resultados de las simulaciones para flujo emergente


\chapter{Discusi\'on y conclusiones}
Comparaci\'on entre ambos modelos, discusion y resultados
hola Springer Science \& Business Media, Aug 26, 2006

\begin{thebibliography}{9}

\bibitem{loops} 
Markus Aschwanden. 
\textit{Physics of the Solar Corona} 
Springer Science \& Business Media, 2006.

\bibitem{flujoemergente} 
Rees and Seemel.
\textit{Corrigeme} 
Line Formation in an Unresolved Magnetic Element: A Test of the Centre of Gravity Method, Astronomy and Astrophysics, 1978



\end{thebibliography}


\end{document}
splay:inline' transform='translate(-341.0002,-807.00001)'>
    <path inkscape:connector-curvature='0' d='m 354.0002,813 -5,5 -5,-5 z' id='path6424' sodipodi:nodetypes='cccc' style='fill:#bebebe;fill-opacity:1;stroke:none'/>
    
  </g>
</svg>
